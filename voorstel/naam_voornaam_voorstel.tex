%=============================================================\cite{Ari2018}=================
% Sjabloon onderzoeksvoorstel bachelorproef
%==============================================================================
% Gebaseerd op LaTeX-sjabloon ‘Stylish Article’ (zie voorstel.cls)
% Auteur: Jens Buysse, Bert Van Vreckem
%
% Compileren in TeXstudio:
%
% - Zorg dat Biber de bibliografie compileert (en niet Biblatex)
%   Options > Configure > Build > Default Bibliography Tool: "txs:///biber"
% - F5 om te compileren en het resultaat te bekijken.
% - Als de bibliografie niet zichtbaar is, probeer dan F5 - F8 - F5
%   Met F8 compileer je de bibliografie apart.
%
% Als je JabRef gebruikt voor het bijhouden van de bibliografie, zorg dan
% dat je in ``biblatex''-modus opslaat: File > Switch to BibLaTeX mode.

\documentclass{voorstel}

\usepackage{lipsum}

%------------------------------------------------------------------------------
% Metadata over het voorstel
%------------------------------------------------------------------------------

%---------- Titel & auteur ----------------------------------------------------

% TODO: geef werktitel van je eigen voorstel op
\PaperTitle{Een vergelijking tussen verschillende webapplicaties in Javascript, Angular en React}
\PaperType{Onderzoeksvoorstel Bachelorproef 2019-2020} % Type document

% TODO: vul je eigen naam in als auteur, geef ook je emailadres mee!
\Authors{Jelle Callewaert\textsuperscript{1}} % Authors
\CoPromotor{TBD\textsuperscript{2}}
\affiliation{\textbf{Contact:}
  \textsuperscript{1} \href{mailto:jelle.callewaert.y9088@student.hogent.be}{jelle.callewaert.y9088@student.hogent.be};
  \textsuperscript{2} \href{}{TBD};
}

%---------- Abstract ----------------------------------------------------------

\Abstract{In deze paper zullen de voor- en nadelen van JavaScript, Angular en React beschreven worden. Er zal onderzocht worden welke technologie voordeliger is in bepaalde situaties. Ook zal de performance gemeten worden en daarna zal die vergeleken worden met elkaar. Dit onderzoek is bedoeld als hulpmiddel voor bedrijven in webontwikkeling om de keuze te maken tussen de technologieën voor de ontwikkeling van webapplicaties. Voor deze bedrijven kan dit helpen de keuze te maken en op die manier minder werk of problemen te hebben. In dit document staat hoe er tewerk werd gegaan en hoe de resultaten bekomen zijn. De verwachtingen zijn een aantal specifieke guidelines voorhet bepalen van welke technologie gebruikt zal worden bij de ontwikkeling van een webapplicatie. Vermoedens zijn dat er toch wel grote verschillen zullen zijn , maar die niet altijd bepalend zullen zijn voor de keuze.
}

%---------- Onderzoeksdomein en sleutelwoorden --------------------------------
% TODO: Sleutelwoorden:
%
% Het eerste sleutelwoord beschrijft het onderzoeksdomein. Je kan kiezen uit
% deze lijst:
%
% - Mobiele applicatieontwikkeling
% - Webapplicatieontwikkeling
% - Applicatieontwikkeling (andere)
% - Systeembeheer
% - Netwerkbeheer
% - Mainframe
% - E-business
% - Databanken en big data
% - Machineleertechnieken en kunstmatige intelligentie
% - Andere (specifieer)
%
% De andere sleutelwoorden zijn vrij te kiezen

\Keywords{Webapplicatieontwikkeling. Angular --- React --- JavaScript} % Keywords
\newcommand{\keywordname}{Sleutelwoorden} % Defines the keywords heading name

%---------- Titel, inhoud -----------------------------------------------------

\begin{document}

\flushbottom % Makes all text pages the same height
\maketitle % Print the title and abstract box
\tableofcontents % Print the contents section
\thispagestyle{empty} % Removes page numbering from the first page

%------------------------------------------------------------------------------
% Hoofdtekst
%------------------------------------------------------------------------------

% De hoofdtekst van het voorstel zit in een apart bestand, zodat het makkelijk
% kan opgenomen worden in de bijlagen van de bachelorproef zelf.
%---------- Inleiding ---------------------------------------------------------

\section{Introductie} % The \section*{} command stops section numbering
\label{sec:introductie}

JavaScript is een van de slechtste programmeertalen. De object prototypen zijn een primitieve en slordige manier om aan OO-programmeren te doen. Die zorgen ervoor dat het niet goed schaalt met grote applicaties. \autocite{Eng2016} 

Technologieën zoals React of Angular hebben de laatste tijd wat interesse gewonnen. React wordt meer en meer gebruikt, maar Angular blijft toch wel populair binnen de business. In full-stack developer vacatures wordt Angular in 59\% van de gevallen genoemd, React slechts in 37\% \autocite{Schlothauer2019}

Dit werk heeft als bedoeling de keuze van technologie voor webapplicaties gemakkelijker te maken. Dit zal vooral voordelig zijn voor bedrijven binnen webdevelopment. De keuze van technologie kan bepalend zijn of een project al dan niet succesvol opgeleverd wordt. We stellen hier de vraag: "Welke eigenschappen van een project bepalen de aan te raden keuze van technologie voor dit project?". Ook zal bepaald worden bij welke eigenschappen een technologie als Angular of React beter is dan puur JavaScript.

%https://medium.com/javascript-non-grata/the-top-10-things-wrong-with-javascript-58f440d6b3d8

%https://hackernoon.com/top-3-most-popular-programming-languages-in-2018-and-their-annual-salaries-51b4a7354e06

%https://medium.freecodecamp.org/a-comparison-between-angular-and-react-and-their-core-languages-9de52f485a76

%---------- Stand van zaken ---------------------------------------------------

\section{State-of-the-art}
\label{sec:state-of-the-art}

JavaScript is altijd al een van de meest gebruikte programmeertalen geweest. JavaScript is een scripttaal die gebruikt wordt voor het interactief maken van een webpagina. De kenmerken van JavaScript worden omschreven als:
\begin{itemize}
	\item Snel
	\item Gemakkelijk
	\item Krachtig
\end{itemize}
\autocite{Garbadge2018}

React is een JavaScript library die het voor de gebruiker gemakkelijk maakt om User Interfaces te maken. Het werd door Facebook ontwikkeld in 2011 met als doel de code voor grote webapplicaties gemakkelijker beheersbaar te maken. \autocite{Chand2019}

Angular is een herschrijving van het AngularJS framework ontwikkeld door Google. De eerste publieke versie (Angular2) werd vrijgegeven in september 2016. AngularJS begon populariteit te verliezen in het voordeel van de nieuwe Angular2. Op 28 mei 2019 werd versie 8 uitgerold. \autocite{Bodrov-Krukowski2018}

Er werden onderzoeken en online artikels gevonden die de verschillen tussen React en Angular benaderen, zoals \textcite{Ari2018}. Ook werd al onderzoek gedaan naar het verschil met Vue.js, een JavaScript framework, maar er zijn geen onderzoeken gevonden over het verschil met pure JavaScript. 

%https://medium.freecodecamp.org/a-comparison-between-angular-and-react-and-their-core-languages-9de52f485a76



% Voor literatuurverwijzingen zijn er twee belangrijke commando's:
% \autocite{KEY} => (Auteur, jaartal) Gebruik dit als de naam van de auteur
%   geen onderdeel is van de zin.
% \textcite{KEY} => Auteur (jaartal)  Gebruik dit als de auteursnaam wel een
%   functie heeft in de zin (bv. ``Uit onderzoek door Doll & Hill (1954) bleek
%   ...'')

%---------- Methodologie ------------------------------------------------------
\section{Methodologie}
\label{sec:methodologie}

Eerst zullen er applicaties ontwikkeld worden met verschillende eigenschappen, zoals grootte, soorten data en complexiteit in JavaScript, Angular en React. Daarna zullen deze applicaties vergeleken worden op basis van bepaalde eigenschappen. 
\paragraph{Snelheid van compilatie}
De snelheid van compilatie is de eerste factor die onderzocht zal worden. De tijd zal gemeten worden en daarna wordt die per technologie vergeleken met elkaar.
\paragraph{Prestatie in runtime}
Voor gebruikers van de webapplicatie is de prestatie in runtime heel belangrijk. Tegenwoordig zijn mensen gewoon van een snelle internetverbinding te hebben, dus moet een webapplicatie ook snel geladen worden. Er zal een backend opgesteld worden waaruit data opgevraagd zal worden. Deze backend is voor alle technologieën dezelfde. Er zal o.a. nagegaan worden hoe snel de data opgehaald kan worden.
\paragraph{Eenvoud}
Om grote applicaties te onderhouden, is eenvoud een van de meest belangrijke aspecten. In dit onderzoek zal bepaald worden hoe gemakkelijk het is om binnen de gebruikte technologie een aanpassing of uitbreiding te maken. Ook zal er nagegaan worden hoeveel lijnen code nodig waren en welke handelingen uitgevoerd moesten worden om tot het eindresultaat te komen.
\paragraph{}
De eigenschappen van de webapplicaties zullen gemeten worden binnen een virtuele machine. De gebruikte software voor de verschillende applicaties in alle drie de technologieën zal Visual Studio Code van Microsoft zijn.
%\begin{table}[h]
%	\begin{tabular}{ll}
%		JavaScript & Brackets \\
%		Angular & Visual Studio Code \\
%		React & Visual Studio Code \\
%	\end{tabular}
%\end{table}

Op basis van de eigenschappen zal dan een overzicht opgesteld worden waarin de technologieën vergeleken zullen worden bij de verschillende applicaties.

%---------- Verwachte resultaten ----------------------------------------------
\section{Verwachte resultaten}
\label{sec:verwachte_resultaten}

Vermoedens zijn dat aan de hand van de resultaten Angular en React heel snel de voorkeur zullen krijgen op JavaScript. Vermoedelijk zullen Angular of React de code overzichtelijker maken, hoewel dat voor een langere compilatietijd zal zorgen. JavaScript zal echter wel nog enkele voordelen metzich meebrengen bij enkele soorten webapplicaties tegenover Angular of React.

%---------- Verwachte conclusies ----------------------------------------------
\section{Verwachte conclusies}
\label{sec:verwachte_conclusies}

De verwachtingen voor dit onderzoek zijn:
\begin{itemize}
	\item JavaScript is sneller bij kleine en gemiddelde applicaties met weinig functionaliteit
	\item Voor webapplicaties met veel functionaliteit en verschillende soorten data zullen Angular en React overzichtelijker zijn.
	\item JavaScript is trager dan Angular of React bij applicaties met veel klassen of interfaces
    \item React zal de snelste zijn wanneer het komt op het aanpassen van het DOM.
	\item Bij grote en complexe applicaties zal JavaScript moeilijker te programmeren zijn
    \item De performance van JavaScript zal in elk van de gevallen beter zijn.
\end{itemize}



%------------------------------------------------------------------------------
% Referentielijst
%------------------------------------------------------------------------------
% TODO: de gerefereerde werken moeten in BibTeX-bestand ``voorstel.bib''
% voorkomen. Gebruik JabRef om je bibliografie bij te houden en vergeet niet
% om compatibiliteit met Biber/BibLaTeX aan te zetten (File > Switch to
% BibLaTeX mode)

\phantomsection
\printbibliography[heading=bibintoc]

\end{document}
