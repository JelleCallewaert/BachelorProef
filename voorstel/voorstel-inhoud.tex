%---------- Inleiding ---------------------------------------------------------

\section{Introductie} % The \section*{} command stops section numbering
\label{sec:introductie}

Volgens Richard Kenneth Eng is JavaScript een van de slechtste programmeertalen, zoals hij beschrijft in zijn artikel \textcite{Richard2016}.  De object prototypen zijn een primitieve en slordige manier om aan OO-programmeren te doen. Die zorgen ervoor dat het niet goed schaalt met grote applicaties. 

Technologieën zoals React of Angular hebben de laatste tijd wat interesse gewonnen, hoewel de interesse in Angular schijnt te dalen. JavaScript blijft echter een van de meest gebruikte programmeertalen. 

Dit werk heeft als bedoeling de keuze van technologie voor bedrijven gemakkelijker te maken. De keuze van technologie kan bepalend zijn of een project al dan niet succesvol opgeleverd wordt. We stellen hier de vraag: "Welke eigenschappen van een project bepalen de aan te raden keuze van technologie voor dit project?". Ook zal bepaald worden welke technologie beter is bij welke eigenschappen.

%https://medium.com/javascript-non-grata/the-top-10-things-wrong-with-javascript-58f440d6b3d8

%https://hackernoon.com/top-3-most-popular-programming-languages-in-2018-and-their-annual-salaries-51b4a7354e06

%https://medium.freecodecamp.org/a-comparison-between-angular-and-react-and-their-core-languages-9de52f485a76

%---------- Stand van zaken ---------------------------------------------------

\section{State-of-the-art}
\label{sec:state-of-the-art}

Volgens het artikel \autocite{Hackernoon2018} is JavaScript altijd al een van de meest gebruikte programmeertalen geweest. JavaScript is een scripttaal die gebruikt wordt voor het interactief maken van een webpagina. De kenmerken van JavaScript worden omschreven als:
\begin{itemize}
	\item Snel
	\item Gemakkelijk
	\item Krachtig
\end{itemize}

React is een JavaScript library die het voor de gebruiker gemakkelijk maakt om User Interfaces te maken. Het werd door Facebook ontwikkeld in 2011 met als doel de code voor grotewebapplicaties gemakkelijker beheersbaar te maken.Angular is een herschrijving van het AngularJS frameworkontwikkeld door Google. De eerste publieke versie (Angular2) werd vrijgegeven in september 2016. Onlangs (oktober 2018) is Angular 7 vrijgegeven. 

Er werden onderzoeken en artikels gevonden die de verschillen tussen React en Angular benaderen, zoals \autocite{FreeCodeCamp2018}, maar er gingen geen over het verschil met gewoon JavaScript. 

%https://medium.freecodecamp.org/a-comparison-between-angular-and-react-and-their-core-languages-9de52f485a76



% Voor literatuurverwijzingen zijn er twee belangrijke commando's:
% \autocite{KEY} => (Auteur, jaartal) Gebruik dit als de naam van de auteur
%   geen onderdeel is van de zin.
% \textcite{KEY} => Auteur (jaartal)  Gebruik dit als de auteursnaam wel een
%   functie heeft in de zin (bv. ``Uit onderzoek door Doll & Hill (1954) bleek
%   ...'')

%---------- Methodologie ------------------------------------------------------
\section{Methodologie}
\label{sec:methodologie}

Eerst zullen er applicaties ontwikkeld worden met verschillende eigenschappen, zoals grootte, soorten data en complexiteit in JavaScript, Angular en React. Daarna zullen deze applicaties vergeleken worden op basis van bepaalde eigenschappen. 
\begin{itemize}
	\item Snelheid van compilatie
	\item Overzichtelijkheid
	\item Eenvoud
\end{itemize}

De eigenschappen van de webapplicaties zullen gemeten worden of er zal een vragenlijst rondgestuurd worden waarbij verschillende meningen gevraagd zullen worden over bijvoorbeeld overzichtelijkheid, die niet direct gemeten kan worden.De gebruikte software voor de verschillende applicaties in verschillende technologieën zullen zijn:
\begin{table}[h]
	\begin{tabular}{ll}
		JavaScript & Brackets \\
		Angular & Visual Studio Code \\
		React & Visual Studio Code \\
	\end{tabular}
\end{table}

Op basis van de eigenschappen zal dan een overzicht opgesteld worden waarin de technologieën vergeleken zullen worden bij de verschillende applicaties.

%---------- Verwachte resultaten ----------------------------------------------
\section{Verwachte resultaten}
\label{sec:verwachte_resultaten}

Vermoedens zijn dat aan de hand van de resultaten Angular en React heel snel de voorkeur zullen krijgen opJavaScript. Vermoedelijk zullen Angular of React de code overzichtelijker maken, hoewel dat voor een langere compilatietijd zal zorgen. JavaScript zal echter wel nog enkele voordelen metzich meebrengen bij enkele soorten webapplicaties tegenoverAngular of React. De antwoorden op de vragenlijst zullen vermoedelijk beperkt en uiteenlopend zijn, omdat deze afhangen van persoonlijke meningen.

%---------- Verwachte conclusies ----------------------------------------------
\section{Verwachte conclusies}
\label{sec:verwachte_conclusies}

De verwachtingen voor dit onderzoek zijn:
\begin{itemize}
	\item JavaScript is sneller bij kleine en gemiddelde applicaties met weinig functionaliteit
	\item Voor webapplicaties met veel functionaliteit en verschillende soorten data zullen Angular en React overzichtelijker zijn.
	\item JavaScript is trager dan Angular of React bij applicatiesmet veel klassen of interfaces
	\item Bij grote en complexe applicaties zal JavaScript moeilijker te programmeren zijn
\end{itemize}

